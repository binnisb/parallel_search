\documentclass[a4paper,10pt]{article}
\usepackage[utf8]{inputenc}
\usepackage{graphicx}
\usepackage{listings}
\usepackage{caption}
\usepackage{subcaption}

\usepackage{listings}
\usepackage[usenames,dvipsnames]{color}
\usepackage{courier}
  
  % This is the color used for MATLAB comments below
\definecolor{MyDarkGreen}{rgb}{0.0,0.4,0.0}
\definecolor{MyPink}{rgb}{1.0,0.0,0.5}
\definecolor{gray}{rgb}{0.5,0.5,0.5}
\definecolor{MyRed}{rgb}{0.8,0.3,0.36}
% For faster processing, load Matlab syntax for listings
\lstloadlanguages{C}%
\lstset{ %
  language=C,						% the language of the code
  basicstyle=\footnotesize,         % the size of the fonts that are used for the code
  numbers=left,                   	% where to put the line-numbers
  numberstyle=\footnotesize,        % the size of the fonts that are used for the line-numbers
  stepnumber=1,                   	% the step between two line-numbers. If it's 1, each line 
                                  	% will be numbered
  numbersep=5pt,                  	% how far the line-numbers are from the code
  backgroundcolor=\color{white},    % choose the background color. You must add \usepackage{color}
  showspaces=false,               	% show spaces adding particular underscores
  showstringspaces=false,         	%	 underline spaces within strings
  showtabs=false,                 	% show tabs within strings adding particular underscores
  frame=single,                   	% adds a frame around the code
  rulecolor=\color{black},        	% if not set, the frame-color may be changed on line-breaks within not-black text (e.g. commens (green here))
  tabsize=2,                      	% sets default tabsize to 2 spaces
  captionpos=b,                   	% sets the caption-position to bottom
  breaklines=true,                	% sets automatic line breaking
  breakatwhitespace=true,        	% sets if automatic breaks should only happen at whitespace
  title=\lstname,                   % show the filename of files included with \lstinputlisting;
                                  	% also try caption instead of title
  numberstyle=\tiny\color{gray},    % line number style
  keywordstyle=\color{MyDarkGreen},        % keyword style
  commentstyle=\color{blue},       	% comment style
  stringstyle=\color{MyPink},       % string literal style
  escapeinside={\%*}{*)},           % if you want to add a comment within your code
  morekeywords={0,1,2,3,4,5,6,7,8,9}            	% if you want to add more keywords to the set
  }

\lstset{
emph={unsigned,char,double,int},emphstyle=\color{MyDarkGreen},
emph={[2]if,for,else,while,return},emphstyle={[2]\color{Red}},
emph={\#include},emphstyle=\color{MyDarkGreen},
emph={0,1,2,3,4,5,6,7,8,9},emphstyle=\color{MyPink}
}










% Title Page
\title{}
\author{}


\begin{document}

\begin{titlepage}
\begin{center}


\textsc{\LARGE Royal Institute of Technology}\\[1.5cm]

\includegraphics[width=0.3\textwidth]{kth_mathematics_rgb.jpg}\\[1cm]

\textsc{\Large Introduction to High-Performance Computing, DN2258 }  %\\ Instructor: Tobias Ryden , \emph{tryd@math.kth.se}}\\[0.5cm]

\hrulefill \\[0.4cm]
{ \huge \bfseries Final project, parallel search\\ Draft}\\[0.4cm]
\hrulefill \\[1.5cm]


Sindri Magnússon, sindrim@kth.se, 871209-7156 \\
Brynjar Smári Bjarnasson, bsbj@kth.se, 840824-4690 \\


\vfill
% Bottom of the page
{\large \today}

\end{center}
\end{titlepage}


%%%%%%%%%%%%%%%%%%%%%%%%%%%%%%%%%%%%%%%%%%%%%%%%%%%%%%%%%%%%%%%%%%%%%%%%%%%%%%%%
%%%%%%%%%%%%%%%%%%%%%%%%%%%%%%%%%%%%%%%%%%%%%%%%%%%%%%%%%%%%%%%%%%%%%%%%%%%%%%%%
%%%%%%%%%%%%%%%%%%%%%%%%%%%%%%%%%%%%%%%%%%%%%%%%%%%%%%%%%%%%%%%%%%%%%%%%%%%%%%%%
%%%%%%%%%%
%%%%%%%%%% Report statrs
%%%%%%%%%%
\cleardoublepage
\tableofcontents
\newpage

\section{Introduction}

In today's scientific environment the amount of data that is generated is increasing like never before.  
One consequence of these revolutionary time is the ability to process and work with the data in acceptable time.  
We plan to look further into one of the most important problem in the data universe which is searching.  



embarrasly parallel


\subsection{The data}
To isolate the main problem and to be able to analyse the results in as derect way as possible we decided to use as simple data as we could think of.
Anohter benefit of a simple data set is the ability of creating a data with a large number of rows. 
The data set is a file containing in every line an id and a sequence of six uniform random integers on the interval $[0,9]$.  
All of the id's and all the sequences of random numbers have the same number of characters so the number of characters in each line is fixed.

Here below we se an example of the data with $1000$ rows.
\lstinputlisting[language=C]{example_data.txt}  
The data we used for testing contains $10^{9}$ lines and each line has $17$ characters.  Since a character is one byte
the file is $17 \times 10^9$ bytes ($17$ Gbyte).


\section{Theoretical performance estimation}

Since each comparison is independent we see that the problem of searching is actually embarrassingly parallel.  
However we need to read the data file from disk which is a sequential operation.  
So if we only look at the searching part of the algorithm then we would expect a linear speedup.  
But if we look at the whole thing and let $P$ be the time percent of the search in the parallel  
part then from Amdahl's law we expect the speedup on $N$ threads to be:
$$ S_N = \frac{1}{(1-P)+\frac{P}{N}}$$

\section{Results}

\subsection{OpemMP}
 
Some graphics of the result we got.  We have still not been able to run it on Lindgren but these 
results are from our personal computers.

\begin{figure}
        \centering
        \begin{subfigure}[b]{0.6\textwidth}
                \centering
                \includegraphics[width=\textwidth]{../graphics/speedup_all.png}
                \caption{The red-line is the theoretical speed-up from Ahmads-law, the stars are
                         show the acctual speed-up.}
                \label{fig:gull}
        \end{subfigure}%
        \\ %add desired spacing between images, e. g. ~, \quad, \qquad etc. 
          %(or a blank line to force the subfigure onto a new line)
        \begin{subfigure}[b]{0.6\textwidth}
                \centering
                \includegraphics[width=\textwidth]{../graphics/speedup_search.png}
                \caption{Speed up of the search part}
                \label{fig:tiger}
        \end{subfigure}
\end{figure}


\subsection{MPI}

%%%%%%%%%%%%%%%%%%%%%%%%%%%%%%%%%%%%%%%%%%
%%%%%%%%%%%%%%%%%%%%%%%%%%%%%%%%%%%%%%%%%%
%%%%%%
%%%%%%  Appendix  
%%%%%%
\appendix
\section{Code}
%  \subsection{serial code}
%    \lstinputlisting[language=C]{../src/search_serial.c}
  \subsection{OpenMP}
    \lstinputlisting[language=C]{../src/search_openmp.c}   
\end{document}          
